\documentclass[12pt,a4paper]{article}
\usepackage[top=2.5cm,bottom=2.5cm,left=2cm,right=2cm]{geometry}
\usepackage[english,romanian]{babel}
\usepackage{amsmath}
\usepackage{amssymb}
\usepackage{amsfonts}
\usepackage{amsthm}
\usepackage{hyperref}

% Definim teoremele folosind comenzi standard
\newtheorem{teorema}{Teorema}
\newtheorem{definitia}{Definiția}
\newtheorem{lema}{Lema}
\newtheorem{propozitia}{Propoziția}

\begin{document}

\title{Exemplu Funcțional LaTeX}
\author{Numele Tău \\
\small{Student, Universitatea, Orașul, Email}}
\date{\today}
\maketitle

\section{Introducere}
Acesta este un exemplu simplu de document LaTeX care funcționează fără erori. Nu trebuie să te îngrijorezi de formatare - LaTeX se ocupă de totul automat.

\section{Matematică Simplă}
Iată câteva exemple de formule matematice:

\subsection{Formule Inline}
Când scrii $x^2 + y^2 = z^2$ în text, LaTeX o formatează automat frumos.

\subsection{Formule Centrate}
Pentru formule importante, folosești dublu dolar:
$$\int_{0}^{1} x^2 dx = \frac{1}{3}$$

Sau cu pătrate:
$$\sum_{i=1}^{n} i = \frac{n(n+1)}{2}$$

\section{Teoreme și Definiții}
\subsection{Definiție}
\begin{definitia}
Un număr prim este un număr natural mai mare decât 1 care nu are divizori pozitivi în afară de 1 și el însuși.
\end{definitia}

\subsection{Teoremă}
\begin{teorema}
Există o infinitate de numere prime.
\end{teorema}

\subsection{Lemă}
\begin{lema}
Dacă $n$ este un număr natural, atunci $n^2 + n + 41$ este prim pentru $n = 0, 1, 2, \ldots, 39$.
\end{lema}

\section{Liste}
\subsection{Listă Numerotată}
\begin{enumerate}
\item Primul element
\item Al doilea element
\item Al treilea element
\end{enumerate}

\subsection{Listă cu Buline}
\begin{itemize}
\item Un element
\item Alt element
\item Încă un element
\end{itemize}

\section{Matematică Avansată}
\subsection{Fracții și Radicali}
$$\frac{a + b}{c + d} = \frac{\sqrt{a^2 + b^2}}{\sqrt{c^2 + d^2}}$$

\subsection{Integrale și Sume}
$$\int_{-\infty}^{\infty} e^{-x^2} dx = \sqrt{\pi}$$

$$\sum_{k=1}^{n} k^2 = \frac{n(n+1)(2n+1)}{6}$$

\subsection{Litere Grecești}
$$\alpha + \beta = \gamma$$
$$\theta \in [0, 2\pi]$$

\section{Concluzie}
LaTeX este foarte simplu de folosit! Trebuie să înveți doar câteva comenzi de bază:
\begin{itemize}
\item \texttt{\textbackslash section\{...\}} pentru secțiuni
\item \texttt{\textbackslash begin\{teorema\}...\textbackslash end\{teorema\}} pentru teoreme
\item \texttt{\$...\$} pentru formule inline
\item \texttt{\$\$...\$\$} pentru formule centrate
\end{itemize}

\section{Exercițiu}
Încearcă să modifici acest document:
\begin{enumerate}
\item Schimbă titlul
\item Adaugă o nouă secțiune
\item Scrie o formulă matematică simplă
\item Compilează din nou
\end{enumerate}

\end{document}
