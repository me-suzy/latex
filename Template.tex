\documentclass[journal,noindent,headline,twoside,indent,a4paper,12pt]{paper_RDMI}
\usepackage[top=2.5cm,bottom=2.5cm,left=2cm,right=2cm,a4paper]{geometry} % Page margins
\usepackage[english,latin,romanian]{babel}
\usepackage{pdfpages}
\usepackage{microtype}
\usepackage{xcolor}
\definecolor{ocre}{RGB}{243,102,25} % Define the orange color used for highlighting throughout the book
\usepackage{lipsum} % for dummy text only
\usepackage[colorlinks,linkcolor=blue!50!black]{hyperref}
\usepackage{graphicx}
%\usepackage[T1]{fontenc}
%\usepackage[utf8]{inputenc}
\usepackage{amsfonts}
\usepackage{amssymb}
\usepackage{amsmath}
\usepackage{latexsym}
\usepackage{amsthm}
\usepackage{verbatim}
\usepackage{extarrows}
\usepackage{combelow}
\usepackage{pgf-pie}
%\usepackage{titling}
\usepackage{tikz}
\usepackage{wrapfig}
\usepackage{caption}
\usepackage{lipsum}
\usepackage{hyperref}
\usepackage{bookmark}
\usepackage[vlined]{algorithm2e}
%\usepackage{titlesec}

\newcommand{\w}{\O}
\renewcommand{\emptyset}{\O}
\newcommand{\ov}{\overline}
\newcommand{\grad}{\mathrm{grad\,}}
\newcounter{p}%[chapter]
\setcounter{p}{0}
\renewcommand{\thep}{\arabic{p}.}
\newenvironment{p}{
\par\bigskip\noindent
\refstepcounter{p}
{\bf\thep}
\bgroup
}{
%\rule{0.5em}{0.5em}
\egroup %\par\bigskip
}
%L
%A
\newcommand{\modul}[1]{\bigl|{#1}\bigr|}
\newcommand{\sumM}[1]{\begin{displaystyle}\sum#1\end{displaystyle}}
%A
\newcommand{\fract}[2]{\displaystyle\frac{#1}{#2}}
\newcommand{\sums}{\displaystyle\sum}
\newcommand{\lil}[1]{\lim\limits_{#1}}
\newcommand{\code}[1]{\texttt{#1}}
\newcommand{\Q}{\mathbb Q}
\newcommand{\K}{\mathbb K}
\newcommand{\R}{\mathbb R}
\newcommand{\N}{\mathbb N}
\newcommand{\Z}{\mathbb Z}
\newcommand{\C}{\mathbb C}
\newcommand{\Mc}{\mathcal{M}}
\newcommand{\A}{\mathcal{A}}
\newcommand{\Mod}{\mathrm{mod\;}}
\newcommand{\Rang}{\mathrm{rang\,}}
\newcommand{\Card}{\mathrm{card\,}}
\newcommand{\lp}{\left(}
\newcommand{\rp}{\right)}
\newcommand{\mo}[1]{\left|{#1}\right|}
\newcommand{\Int}{\textrm{Int}\,}
\newcommand{\tg}{\mathrm{tg\,}}
\newcommand{\ctg}{\mathrm{ctg\,}}
\newcommand{\arctg}{\mathrm{arctg\,}}
\newcommand{\re}{\mathrm{Re\,}}
\newcommand{\im}{\mathrm{Im\,}}
\newcommand{\integ}{\displaystyle\int}
\newcommand{\OM}{\mathcal{O}\,}
\newcommand{\card}{\text{\rm card}}
\newcommand{\intl}[2]{\int\limits_{#1}^{#2}}
\newcommand{\tr}{\mathrm{tr\;}}

\newtheorem{defi}{Defini\cb{t}ia}
\newtheorem{teo}{Teorema}
\newtheorem{prop}{Propozi\cb{t}ia}
\newtheorem*{prop*}{Propozi\cb tie}
\newtheorem{app}{Aplica\cb tia}
\newtheorem{pbR}{\bf{M}}                           % PROBLEMA REZOLVATA
\newtheorem{probl}{Problema}
\newtheorem{exc}{Exerci\cb tiul}
%\renewcommand{\proofname}{Solu\cb{t}ie}
\newenvironment{sol}{\noindent{\it{Solu\cb tie. }}}{}
\renewcommand{\refname}{BIBLIOGRAFIE}
\renewcommand{\proofname}{Demonstra\cb{t}ie}
\newtheorem{apl}{Aplica\cb{t}ia}
\newtheorem{cor}{Corolarul}
\newtheorem{lema}{Lema}

\theoremstyle{remark}
\newtheorem{obs}{Observa\cb{t}ia}
\newtheorem{ex}{Exemplul}
\newtheorem{pb}{\bf{M}}
\newtheorem{pbi}{\bf{I}}
\newtheorem{pbo}{\bf{O}}
\newenvironment{solu}{\vspace{-10pt}\noindent{\bf Solu\cb tie:}}{\hfill$\square$}                         % PROBLEMA PROPUSA


\newcommand{\ResetContoare}
{
\setcounter{defi}{0}
\setcounter{teo}{0}
\setcounter{prop}{0}
\setcounter{app}{0}
\setcounter{probl}{0}
\setcounter{exc}{0}
\setcounter{obs}{0}
\setcounter{figure}{0}
}

\begin{document}
\title{Titlu articol}
\shorttitle{Titlu scurt pentru antet (dacă este cazul a fi scurtat}
\author{Prenume Nume
\thanks{Profesor, Unitatea de învățământ, Localitatea, E-mail}}
\shortauthor{Inițiala Prenume. Nume}
\institution{}
\date{\today}
\maketitle

\section{Reguli importante de scriere}
Textul va fi scris în Word normal la un rând cu Times New Roman, 12 pt.
Toate variabilele și relațiile matematice se vor scrie în modul equation/mathtype $n\ge 4$, iar în LaTex între 2 dolari - $a_{1} ,a_{2} ,\ldots ,a_{n} \ge 0$.
Relațiile importante vor fi centrate pe rând nou - Exemple:
$$\mathop{\sum }\limits_{1\le i<j\le n} a_{i} a_{j} =\frac{n\left(n-1\right)}{2}.$$
sau
\[\sqrt{\frac{n-2}{n} } \cdot \left(a_{1} +a_{2} +\ldots +a_{n} \right)+\frac{\left(\sqrt{n} -\sqrt{n-2} \right)^{2} }{2} \cdot a_{1} a_{2} \ldots a_{n} \ge n-1.\]
\section{Alte  exemple}
Enumerare:
\begin{enumerate}
\item $a_{1} +a_{2} +\ldots +a_{n} \ge n.$
\item If $a_{1} ,a_{2} ,\ldots ,a_{n}$  are real numbers satisfying $\mathop{\sum }\limits_{1\le i<j\le n} a_{i} a_{j} =\frac{n\left(n-1\right)}{2}$  and
$n\le a_{1} +a_{2} +\ldots +a_{n} <\left(n-1\right)\sqrt{\frac{n}{n-2} }$, then all of them are strictly positive.
\item If $a_{1} ,a_{2} ,\ldots ,a_{n}$  are real numbers satisfying $\mathop{\sum }\limits_{1\le i<j\le n} a_{i} a_{j} =\frac{n\left(n-1\right)}{2}$  and
$a_{1} +a_{2} +\ldots +a_{n} \ge n$, then there exists $t\ge 1$ such that $a_{1} +a_{2} +\ldots +a_{n} =n\left(\frac{t^{2} +1}{2t} \right)$.
\end{enumerate}
Teoreme, Leme, Propoziții ... :
\begin{lema}
Let ... be a fixed real number. We consider the positive real numbers $a_{1} ,a_{2} ,\ldots ,a_{n} $
satisfying ... .
Then $\min \left(a_{1} a_{2} \ldots a_{n} \right)=\frac{t^{n-2} \left[n-\left(n-2\right)t^{2} \right]}{2}.$
\end{lema}

\begin{thebibliography}{9}
\bibitem{1} (Autor) Inițiala prenume. Nume, \textit{Titlu}, Sursa, Editura, (an), (pagini) 25-26.
\end{thebibliography}


\end{document} 