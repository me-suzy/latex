\documentclass[journal,noindent,headline,twoside,indent,a4paper,12pt]{paper_RDMI}
\usepackage[top=2.5cm,bottom=2.5cm,left=2cm,right=2cm,a4paper]{geometry}
\usepackage[english,latin,romanian]{babel}
\usepackage{amsmath}
\usepackage{amssymb}
\usepackage{amsfonts}

\begin{document}

\title{Exemplu Simplu LaTeX}
\shorttitle{Exemplu Simplu}
\author{Numele Tău
\thanks{Student, Universitatea, Orașul, Email}}
\shortauthor{N. Tău}
\institution{}
\date{\today}
\maketitle

\section{Introducere}
Acesta este un exemplu simplu de document LaTeX. Nu trebuie să te îngrijorezi de formatare - LaTeX se ocupă de totul automat.

\section{Matematică Simplă}
Iată câteva exemple de formule matematice:

\subsection{Formule Inline}
Când scrii $x^2 + y^2 = z^2$ în text, LaTeX o formatează automat frumos.

\subsection{Formule Centrate}
Pentru formule importante, folosești dublu dolar:
$$\int_{0}^{1} x^2 dx = \frac{1}{3}$$

Sau cu pătrate:
$$\sum_{i=1}^{n} i = \frac{n(n+1)}{2}$$

\section{Teoreme și Definiții}
\subsection{Definiție}
\begin{defi}
Un număr prim este un număr natural mai mare decât 1 care nu are divizori pozitivi în afară de 1 și el însuși.
\end{defi}

\subsection{Teoremă}
\begin{teo}
Există o infinitate de numere prime.
\end{teo}

\section{Liste}
\subsection{Listă Numerotată}
\begin{enumerate}
\item Primul element
\item Al doilea element
\item Al treilea element
\end{enumerate}

\subsection{Listă cu Buline}
\begin{itemize}
\item Un element
\item Alt element
\item Încă un element
\end{itemize}

\section{Concluzie}
LaTeX este foarte simplu de folosit! Trebuie să înveți doar câteva comenzi de bază:
\begin{itemize}
\item \texttt{\textbackslash section\{...\}} pentru secțiuni
\item \texttt{\textbackslash begin\{teo\}...\textbackslash end\{teo\}} pentru teoreme
\item \texttt{\$...\$} pentru formule inline
\item \texttt{\$\$...\$\$} pentru formule centrate
\end{itemize}

\end{document}
